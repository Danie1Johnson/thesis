\section{Constrained Dynamics}

Sometimes we are interested in more than simply the number of degrees of freedom a particular configuration has. To explore the configuration space, find similarities between different configurations, analyze the geometric configuration space's topology, or compute other relevant statistics, it can be useful to compute dynamic processes on the constraint space. There are clear computational challenges to computing such dynamics due to computational handling of constraints, the lack of an explicit parameterizations, and the fact that geometric configuration space is often of much smaller dimension than the ambient space it sits in.

Constraint algorithms have been widely studied for use in molecular dynamic simulations. Like many of the existing methods, we use a two stage scheme in which the unconstrained dynamics are first used to evolve the system to an intermediary configuration $\hat{z}_{n+1}$ based on the previous configuration $z_n$. In the second stage, Lagrange multipliers are used to correct $\hat{z}_{n+1}$ to a new configuration $z_{n+1}$ which satisfies the constraints. Previously, we used the SHAKE algorithm--which employs this two stage philosophy--to examine a constraint space used to model the configurations of cyclohexane CITE. 

Assuming connectedness of the configuration space, constraint algorithms can be used to compute various statistics of the configuration space. For example, the $d$-dimensional volume of a $d$-dimensional configuration space can be used to measure how much mobility the intermediate has in its configuration space. Additionally, if we have soft constraints, such as preferred angles between faces, we can institute cost functions on the configuration space with configurations having more favorable angles having a lower cost. Whether physically or artificially motivated, such cost functions can be treated as potential energy functions which can imply specific dynamics. Under this set of dynamics, quantities such as the average or minimum energies may be of interest. 

We may also be interested in the behavior of a small part of the configuration as it undergoes constrained dynamics. For instance, suppose we are interested in the tendency for two edges on distinct faces to come together and form a new connection. By looking at the distance between these edges, we can measure how frequently this event happens and the probability that it will occur before another event of interest. This is similar to the exit time problem for diffusion and may provide a physically motivated evaluation tool for the appropriateness of our transition rules we adopt in our models. 

\subsection{Cyclohexane Dynamics}

??include??

%\label{introduction}
%Cyclohexane is a molecule composed of six carbon atoms and twelve hydrogen atoms. The carbon atoms are connected in a ring with two hydrogen atoms attaching to each carbon. Since each carbon has four bonds, the energetically preferred bond spacing is at tetrahedral angles. While this preference dictates much of the cyclohexane structure, there are several structurally distinct conformations the molecule can take. 
%
%Of the many forces acting on the cyclohexane molecule, we focus on three. \textbf{Eclipsing strain} refers to the force between the carbon atoms that prevents them from getting too close to each other. This imposes a preferred distance between each pair of bonded atoms. Second, \textbf{angle strain} corresponds to the carbon atom's four bonds trying to spread apart from each other. As mentioned before, tetrahedral angles are preferred. Finally, \textbf{steric crowding} is similar to eclipsing strain, but the eclipsing in this case is between the hydrogen atoms bonded to the carbons.
%
%\subsubsection{Configurations}
%
%Due to the aforementioned forces, there are a variety of frequently observed configurations with geometries that alleviate these strains to varying amounts. 
%
%The lowest energy configuration is the \textbf{chair}. With the absence of both angle and eclipsing strain, the chair only has a small amount of steric strain. 
%
%Another well studied configuration is the \textbf{boat}. With two sets of parallel carbons arranged in a planar rectangular structure, the remaining two carbon atoms sit at opposite ends slightly above the plain. The structure resembles the bow and stern of a boat. Despite having little angle strain, it does have some steric crowding between the hydrogen atoms at the ends of the boat as well as some eclipsing stain.
%
%
%
%\begin{figure}
%        %\centering
%%        \begin{subfigure}[b]{0.24\textwidth}
%%                \includegraphics[width=\textwidth]{pv_chair_1.png}
%%                %\caption{Chair}
%%                %\label{fig:Chair}
%%        \end{subfigure}%
%%        ~ %add desired spacing between images, e. g. ~, \quad, \qquad etc.
%%          %(or a blank line to force the subfigure onto a new line)
%%        \begin{subfigure}[b]{0.24\textwidth}
%%                \includegraphics[width=\textwidth]{pv_boat_1.png}
%%                %\caption{Boat}
%%                %\label{fig:Boat}
%%        \end{subfigure}%
%%        ~ %add desired spacing between images, e. g. ~, \quad, \qquad etc.
%%          %(or a blank line to force the subfigure onto a new line)
%%        \begin{subfigure}[b]{0.24\textwidth}
%%                \includegraphics[width=\textwidth]{pv_twist_boat_2.png}
%%                %\caption{Twist Boat}
%%                %\label{fig:TwistBoat}
%%        \end{subfigure}%
%%        ~ %add desired spacing between images, e. g. ~, \quad, \qquad etc.
%%          %(or a blank line to force the subfigure onto a new line)
%%        \begin{subfigure}[b]{0.24\textwidth}
%%                \includegraphics[width=\textwidth]{pv_twist_chair_2.png}
%%                %\caption{Twist Chair}
%%                %\label{fig:TwistChair}
%%        \end{subfigure}%
%%
%%        \begin{subfigure}[b]{0.24\textwidth}
%%                \includegraphics[width=\textwidth]{pv_chair_2.png}
%%                %\caption{Chair}
%%                %\label{fig:Chair}
%%        \end{subfigure}%
%%        ~ %add desired spacing between images, e. g. ~, \quad, \qquad etc.
%%          %(or a blank line to force the subfigure onto a new line)
%%        \begin{subfigure}[b]{0.24\textwidth}
%%                \includegraphics[width=\textwidth]{pv_boat_2.png}
%%                %\caption{Boat}
%%                %\label{fig:Boat}
%%        \end{subfigure}%
%%        ~ %add desired spacing between images, e. g. ~, \quad, \qquad etc.
%%          %(or a blank line to force the subfigure onto a new line)
%%        \begin{subfigure}[b]{0.24\textwidth}
%%                \includegraphics[width=\textwidth]{pv_twist_boat_1.png}
%%                %\caption{Twist Boat}
%%                %\label{fig:TwistBoat}
%%        \end{subfigure}%
%%        ~ %add desired spacing between images, e. g. ~, \quad, \qquad etc.
%%          %(or a blank line to force the subfigure onto a new line)
%%        \begin{subfigure}[b]{0.24\textwidth}
%%                \includegraphics[width=\textwidth]{pv_twist_chair_1.png}
%%                %\caption{Twist Chair}
%%                %\label{fig:TwistChair}
%%        \end{subfigure}%
%%
%%        \begin{subfigure}[b]{0.24\textwidth}
%%                \includegraphics[width=\textwidth]{pv_chair_3.png}
%%                \caption{Chair}
%%                \label{fig:Chair}
%%        \end{subfigure}%
%%        ~ %add desired spacing between images, e. g. ~, \quad, \qquad etc.
%%          %(or a blank line to force the subfigure onto a new line)
%%        \begin{subfigure}[b]{0.24\textwidth}
%%                \includegraphics[width=\textwidth]{pv_boat_3.png}
%%                \caption{Boat}
%%                \label{fig:Boat}
%%        \end{subfigure}%
%%        ~ %add desired spacing between images, e. g. ~, \quad, \qquad etc.
%%          %(or a blank line to force the subfigure onto a new line)
%%        \begin{subfigure}[b]{0.24\textwidth}
%%                \includegraphics[width=\textwidth]{pv_twist_boat_3.png}
%%                \caption{Twist Boat}
%%                \label{fig:TwistBoat}
%%        \end{subfigure}%
%%        ~ %add desired spacing between images, e. g. ~, \quad, \qquad etc.
%%          %(or a blank line to force the subfigure onto a new line)
%%        \begin{subfigure}[b]{0.24\textwidth}
%%                \includegraphics[width=\textwidth]{pv_twist_chair_3.png}
%%                \caption{Twist Chair}
%%                \label{fig:TwistChair}
%%        \end{subfigure}%
%%
%%
%%        \caption{Cyclohexane Configurations}\label{fig:animals}
%\end{figure}
%
%
%Besides the chair and boat, there are a few intermediate configurations that cyclohexane assumes as when it transitions between the chair and boat. The aptly named \textbf{twist boat} is similar to the boat, but some of the carbons are rotated from the rest of the molecule. Due to this twisting, the twist boat actually has a lower energy than the boat as the hydrogen atoms at the bow and stern are no longer aligned. Interestingly, the boat can transition via the twist boat to a second boat configuration in which the bow and stern atoms are different, but with an otherwise indistinguishable geometry. As part of the boat's transition to the chair, the \textbf{twist chair} is another distinguished configuration. Having large angle and eclipsing strains, the twist boat represents somewhat of an energy barrier between the chair and boat.  
%
%
%\subsubsection{Finding Intermediate Coordinates}
%
%Since any configuration has an infinitude of equivalent configurations given by rotations and translations, we fix the first three atoms $v_1, v_2,v_3$, at positions $c_1, c_2, c_3$ that satisfy the length and angle constraints. These account for nine of the twenty-one total constraint equations. 
%
%We wish to examine each of the four cyclohexane configurations to verify if they satisfy these constraints, but to do so, an explicit representation of the coordinates of each configuration is required. Since we know from Sachse's model that the boat and chair have a special polyhedral representation, geometry enables us to find such coordinates. We pick the $c_1 = \left(-\sqrt{\frac{2}{3}},0,\sqrt{\frac{1}{3}}\right)$, $c_2 = \left(0,0,0\right)$, $c_3  = \left(\sqrt{\frac{2}{3}},0,\sqrt{\frac{1}{3}}\right)$, and solve for the following coordinates.
%
%\begin{align*}
%v_1^{(chair)} &= \left(-\sqrt{\frac{2}{3}},0,\sqrt{\frac{1}{3}}\right) &= v_1^{(boat)} &= \left(-\sqrt{\frac{2}{3}},0,\sqrt{\frac{1}{3}}\right) \\
%v_2^{(chair)}  &= \left(0,0,0\right) &= v_2^{(boat)}  &= \left(0,0,0\right) \\
%v_3^{(chair)}  &= \left(\sqrt{\frac{2}{3}},0,\sqrt{\frac{1}{3}}\right) &= v_3^{(boat)}  &= \left(\sqrt{\frac{2}{3}},0,\sqrt{\frac{1}{3}}\right) \\
%v_4^{(chair)}  &= \left(\sqrt{\frac{2}{3}},\sqrt{\frac{2}{3}},2\sqrt{\frac{1}{3}}\right) &= v_4^{(boat)}  &= \left(\sqrt{\frac{2}{3}},\sqrt{\frac{2}{3}},2\sqrt{\frac{1}{3}}\right) \\
%v_5^{(chair)}  &= \left(0,\sqrt{\frac{2}{3}},\sqrt{3}\right) & v_5^{(boat)}  &= \left(0,\frac{5}{3}\sqrt{\frac{2}{3}},\frac{5}{3}\sqrt{3}\right) \\
%v_6^{(chair)}  &= \left(-\sqrt{\frac{2}{3}},0,\sqrt{\frac{1}{3}}\right) &= v_6^{(boat)}  &= \left(-\sqrt{\frac{2}{3}},0,\sqrt{\frac{1}{3}}\right)
%\end{align*}
%%\begin{align*}
%%v_1^{(chair)} &= \left(-\sqrt{\frac{2}{3}},0,\sqrt{\frac{1}{3}}\right) & v_4^{(chair)}  &= \left(\sqrt{\frac{2}{3}},\sqrt{\frac{2}{3}},2\sqrt{\frac{1}{3}}\right) \\
%%v_2^{(chair)}  &= \left(0,0,0\right) & v_5^{(chair)}  &= \left(0,\sqrt{\frac{2}{3}},\sqrt{3}\right) \\
%%v_3^{(chair)}  &= \left(\sqrt{\frac{2}{3}},0,\sqrt{\frac{1}{3}}\right) & v_6^{(chair)}  &= \left(-\sqrt{\frac{2}{3}},0,\sqrt{\frac{1}{3}}\right) \\
%%v_1^{(boat)} &= \left(-\sqrt{\frac{2}{3}},0,\sqrt{\frac{1}{3}}\right) & v_4^{(boat)}  &= \left(\sqrt{\frac{2}{3}},\sqrt{\frac{2}{3}},2\sqrt{\frac{1}{3}}\right) \\
%%v_2^{(boat)}  &= \left(0,0,0\right) & v_5^{(boat)}  &= \left(0,\sqrt{\frac{2}{3}},\sqrt{3}\right) \\
%%v_3^{(boat)}  &= \left(\sqrt{\frac{2}{3}},0,\sqrt{\frac{1}{3}}\right) & v_6^{(boat)}  &= \left(-\sqrt{\frac{2}{3}},0,\sqrt{\frac{1}{3}}\right)
%%\end{align*}
%
%It is easily verified that both the boat and chair coordinates satisfy all of the constraint equations exactly. As for the twist boat and twist chair, we do not have a precise definition of their coordinates other than they exist somewhere in transition between the boat and chair.  
%
%\subsubsection{Dynamics}
%
%As we are interested in transitions between the four configurations, we can use our constraint model to examine such movements. Is it possible to start with the chair coordinates and continuously deform them to the boat coordinates without ever breaking any of the constraints? Is there a similar deformation between different boat configurations that satisfies the constraints? If so, how hard is it to find such a path?
% 
%
%\subsubsection{Transitioning Between Boat Configurations}
%
%Under the conjecture that it is indeed possible to transition between two boat configurations, some numerical and visual experimentation was used to find coordinates to a second boat configuration, $\textbf{boat2}$. Even though it was easily verified that boat2 satisfies all of the constraints and was equivalent to the first boat by translation and rotation up to a permutation of the carbon atoms, it was not clear how to find a path between the two boats. Common in molecular dynamics simulations, a constrained dynamics SHAKE-type scheme was used to explore the admissible paths away from the boat configuration.
%
%The method is a two stage scheme in which we first step to a new configuration in the direction of boat2 and second enforce the constraints with Lagrange multipliers to get the updated configuration. For mathematical simplicity, we represent the coordinates $v_1,\dots,v_6$ as a single point $x \in \mathbbm{R}^{18}$ where $\left(v_j\right)_k = x_{3j+k}$. 
%
%It is very important to make the estimate $\hat{x}^{n+1}$ of the next configuration satisfy the constraints reasonably well, as it will make the correction step easier. Rather than just defining the naive update of
%$$\hat{x}^{n+1} = x^n + \frac{1}{2}\left(\Delta t\right)^2\left(x^{boat2}-x^n\right),$$
%we seek a smarter method for making the estimate $\hat{x}^{n+1}$. Since we know the that the boat has one degree of freedom, its null space has one dimension. If we step in the direction of null-space $\nu^n \in \mathbbm{R}^{18}$, the constraints should still be close to being satisfied. Thus, we use the following estimate. 
%$$\hat{x}^{n+1} = x^n + \frac{1}{2}\left(\Delta t\right)^2\left[\nu^n\cdot\left(x^{boat2}-x^n\right)\frac{\nu^n}{\|\nu^n\|}\right]$$
%To enforce our constraints on this prediction, we use Lagrange multipliers. We define the update to be
%$$x^{n+1} = \hat{x}^{n+1} + \left(\Delta t\right)^2\sum^{21}_{k=1}\lambda_k^{n+1}\nabla\varphi_k\left(x^n\right)$$ 
%and find $\lambda^{n+1}$ such that $x^{n+1}$ satisfies the constraint equations. To do this, we plug $x^{n+1}$ into the constraint equations $\varphi$ and use Newton iteration to find $\lambda^{n+1}$.
%
%
%\begin{figure}
%	%\includegraphics[width=\textwidth]{coords_v3.png}
%	\caption{Boat to Boat2 Coordinate Transition}
%	\label{fig:Coords}
%\end{figure}
%
%Using this scheme, we were able to simulate the transition between the boat and boat2 configurations. While maintaining the constraints up to arbitrary precision, the carbon coordinates $x$ were updated along a path that lead from the boat to boat2. This serves as confirmation that it is possible to maintain the constraints given by our model and transition between boat and twist boat configurations. The continuous change in carbon coordinates during this transition can be seen in figure~\ref{fig:Coords}.
%
%This algorithm also enabled us to approximate coordinates for the twist boat as they were taken to be midway between the boat and boat2. Also, by relaxing the constraints to only enforce fixed lengths between carbons and allowing some angle strain, a path from the Chair to the twist boat was found. Similarly, this path allowed for the approximation of the twist chair coordinates.
%
%\subsubsection{Analogy to Folding Model}
%Transition between the different configurations of cyclohexane can be compared to the transitioning within the folding model configuration space. Just as each folding intermediate is represented by a node in a graph, the Cyclohexane intermediates can similarly be organized. In both graphs an edge would mean that the intermediates at either end could transition into each other. As seen in figure~\ref{fig:cyclofold}, to transition between the chair and the boat, the molecule must first become a twist-chair and then a twist boat before it can finally become a boat. Similarly, to transition between the boat and boat2, the twist boat intermediate must first be visited. This relationship is mirrored by the boat and octahedron folding intermediates.
%
%\begin{figure}[h]
%	%\includegraphics[width=\textwidth]{cyclohexane_folding.png}
%	\caption{Configuration Graphs for Cyclohexane and Folding}
%	\label{fig:cyclofold}
%\end{figure}
\section{Manifold Brownian Motion}
\subsection{Computational Scheme}

Suppose we have a $m$ dimensional manifold $\mathcal{M} \subset \mathbbm{R}^n$ defined implicitly by $\mathcal{M} \doteq \left\{x \in \mathbbm{R}^n : c\left(x\right) = 0\right\}$ for some function $c: \mathbbm{R}^n \to \mathbbm{R}^{n-m}$ with $c_1,\dots,c_{n-m}$ independent constraint functions. We seek to approximate a Brownian motion on $\mathcal{M}$ by constructing a random walk. As mentioned before, this is done in two stages; starting at a point $x \in \mathcal{M}$ we take a step of size $\sqrt{\Delta t}$ in a tangent direction $w$ and then we project the point $x +\sqrt{\Delta t} w$ back onto $\mathcal{M}$. 

\subsubsection{Sampling in Tangent Space}  

At each point $z \in \mathcal{M}$ the tangent space is $\mathcal{T}_z\mathcal{M} \doteq \left\{w \in \mathbbm{R}^n : C\left(x\right)w = 0\right\}$ where $C: \mathbbm{R}^n \to \mathbbm{R}^{(n-m)\times n}$ is the Jacobian of $c$. Thus, sampling from the tangent space reduces to sampling from the null space of $C(z)$. 

First we must construct a basis for $\mathcal{T}_z\mathcal{M}$. Let $A = [C^T B]$ be the concatenation of $C^T$ and a randomly drawn matrix $B \in \mathbbm{R}^{n\times m}$. We assume each entry of $B$ is independent with $B_{jk} \sim  \text{unif}(0,1)$. 
\begin{mylem}
The matrix $A$ is of full rank with probability 1. 
\end{mylem}
\begin{proof}
Since the constraints of $c$ are independent, $C$ is of full rank. Additionally, since $B$ is drawn randomly, each column of $B$ will be independent of the columns of $C^T$  and also the other columns of $B$ almost surely. Thus, the columns of $A$ are independent and $A$ is of full rank. 
\end{proof}

Now, we compute the QR decomposition 
\begin{align}
        A = \left[C^T B\right] = QR = \left[Q^{(1)} Q^{(2)}\right] R
\end{align}
where $Q^{(1)} \in \mathbbm{R}^{(n-m)\times n}$ and $ Q^{(2)} \in \mathbbm{R}^{m\times n}$.
\begin{mythm}
The columns of $Q^{(2)}$ are an orthonormal basis for $\mathcal{T}_z\mathcal{M} = N(C)$ and the columns of $Q^{(1)}$ are an orthonormal basis for $(\mathcal{T}_z\mathcal{M})^\perp = (N(C))^\perp$.
\end{mythm}
\begin{proof}
The second claim is true by simple linear algebra arguments and the properties of the QR decomposition.
\begin{align}
        \text{col} (Q^{(1)}) &= \text{col}(C^T)\\
        &= ((\text{col}(C^T))^\perp)^\perp\\
        &= (N((C^T)^T)^\perp \\ 
        &= (N(C))^\perp
\end{align}
Here we use the vector space identities $X = (X^\perp)^\perp$ and $(\text{col} X)^\perp = N(X^T)$. This shows that columns of $Q^{(1)}$ are an orthonormal basis for $(\mathcal{T}_z\mathcal{M})^\perp$. Now, since 
\begin{align}
        \text{col} (Q^{(1)}) \oplus \text{col} (Q^{(2)}) &= \text{col}(A) \\
        &= \mathbbm{R}^n \\
        & = (N(C)) \oplus (N(C))^\perp \\
        & = (N(C)) \oplus \text{col} (Q^{(1)})
\end{align}
 we must have that $\text{col}(Q^{(2)}) = N(C)$ and that the columns of $Q^{(2)}$ provide an orthonormal basis for $\mathcal{T}_z\mathcal{M}$.
\end{proof}
Now, with our orthonormal basis for $\mathcal{T}_z\mathcal{M}$, we can define any $w \in \mathcal{T}_z\mathcal{M}$ by a vector $\alpha \in\mathbbm{R}^m$ as follows.
$$w = \sum_{i=1}^m \alpha_iQ^{(2)}_i = Q^{(2)}\alpha$$
So, for any two $w,\tilde{w} \in \mathcal{T}_z\mathcal{M}$ we can define the metric $g$ as
\begin{align}
        g(w,\tilde{w}) &= g\left(\sum_{i=1}^m \alpha_iQ^{(2)}_i, \sum_{j=1}^m \tilde{\alpha}_jQ^{(2)}_j\right) \\
        &= \sum_{i=1}^m\sum_{j=1}^m  \alpha_i \tilde{\alpha}_j g\left(Q^{(2)}_i, Q^{(2)}_j\right) \\
        &= \sum_{i=1}^m\sum_{j=1}^m  \alpha_i \tilde{\alpha}_j G_{ij} \\
        &= \alpha^T  G\tilde{\alpha}
\end{align}
Then, our goal is to sample uniformly from $\Omega_G = \left\{w = Q^{(2)}\alpha | \alpha^T G\alpha = 1\right\}$. Since this set corresponds to a level set of the Multivariate normal distribution with mean zero and covariance matrix $G^{-1}$, our sampling problem reduces to sampling $u \sim \mathcal{N}(0,G^{-1})$. Given $u$, $w \doteq \frac{u}{|u|}$ is sampled from $\Omega_G$ as desired.We generally choose $G = I$, but this is not required. 

\subsubsection{Projection onto $\mathcal{M}$}
After making a step away from $x$ of size $\sqrt{\Delta t}$ in the direction $w \in N(C)$, our new point $x + \sqrt{\Delta t}w$ is not likely to be in $\mathcal{M}$. To return our point to $\mathcal{M}$, we make a projection step $w^\perp \in (N(C))^\perp$. Since $Q^{(1)}$ provides a basis for $(N(C))^\perp$, we can write any such step as 
$$w^\perp = \sum_{i=1}^{n-m} \gamma_iQ^{(1)}_i = Q^{(1)}\gamma.$$
with $\gamma \in \mathbbm{R}^{n-m}$. Thus, we want to identify a $\gamma$ such that $c(x + \sqrt{\Delta t} w + Q^{(1)}\gamma) = 0$. Since $c$ is nonlinear, we use Newton-Raphson iteration to solve for $\gamma$. By defining an objective function $F$ and its Jacobian $\mathcal{J}$ as
\begin{align}
        F(\gamma) &= c(x + \sqrt{\Delta t} w + Q^{(1)}\gamma) \\
        \mathcal{J}_{jk}(\gamma) &= \frac{\partial F_j}{\partial \gamma_k}(\gamma) \\
        &= \sum_{i=1}^{n-m} \frac{\partial c_j}{\partial x_i}(x + \sqrt{\Delta t} w + Q^{(1)}\gamma)Q^{(1)}_{ik} \\
        \mathcal{J}(\gamma) &= C(x + \sqrt{\Delta t} w + Q^{(1)}\gamma)Q^{(1)}
\end{align}
we arrive at the iteration routine below. 
\begin{align}
        C(x + \sqrt{\Delta t} w + Q^{(1)}\gamma)Q^{(1)}\left(\gamma_{k+1} - \gamma_{k}\right) &= \mathcal{J}(\gamma_k)\left(\gamma_{k+1} - \gamma_{k}\right)\\
         &= -F(\gamma_k) \\
         &= -c(x + \sqrt{\Delta t} w + Q^{(1)}\gamma)
\end{align}
The iteration starts with $\gamma_0 = 0$.

\subsubsection{Sampling Algorithm}

\begin{figure}[ht]
\centering
\begin{algorithmic}
\For{$k=1,\dots,N$}
        \State $B \gets \text{unif}(0,1)^{n\times m}$       
        \State $A \gets \left[C^T B\right]$
        \State $[Q^{(1)}, Q^{(2)}], R \gets$QRdecomposition$(A)$
        \State $\alpha \gets \mathcal{N}(0,G^{-1})$
        \State $w \gets \frac{Q^{(2)}\alpha}{|Q^{(2)}\alpha|}$
        \State $\gamma \gets 0^{n-m}$
        \While{$|c(x + \sqrt{\Delta t}w + Q^{(1)}\gamma)| > \epsilon$}
                     \State $\Delta\gamma \gets (C(x + \sqrt{\Delta t} w + Q^{(1)}\gamma)Q^{(1)})^{-1}(-c(x + \sqrt{\Delta t} w + Q^{(1)}\gamma))$
                     \State $\gamma \gets \gamma + \Delta\gamma$
        \EndWhile   
        \State $x \gets x + \sqrt{\Delta t}w + Q^{(1)}\gamma$
\EndFor
\end{algorithmic}
\caption{Random Walk on $\mathcal{M}$}
\end{figure}


%\begin{figure}[ht]
%\centering
%\begin{algorithmic}
%\end{algorithmic}
%\caption{Algorithm for generating new configuration sample.}
%\label{alg:MBM}
%\end{figure}

\subsection{Validation and Test Cases}
\subsection{MBM on Geometric Configuration Spaces}
\subsubsection{Fixing the Center of Mass}
\subsubsection{Fixing Rotations}

\section{Manifold Reflected Brownian Motion}
\subsection{Computational Scheme}
\subsubsection{Triangle Intersection}
\subsection{Validation and Test Cases}
\subsection{MRBM on Geometric Configuration Spaces}

\section{Computational Implementation}
