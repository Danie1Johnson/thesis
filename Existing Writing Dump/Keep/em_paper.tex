\title{Enumeration of the Building Game}
\author{
        Daniel Johnson \& Govind Menon\\
                Division of Applied Mathematics\\
        Brown University
}
\date{\today}

\documentclass[12pt]{article}

\usepackage{graphicx,amsmath,mathtools,bbm,amsthm,enumerate}
\usepackage{mathrsfs}
%\usepackage[subnum]{cases}

%\usepackage[titletoc,toc,title]{appendix}


\newtheorem{mythm}{Theorem}
\newtheorem{mylem}{Lemma}
\newtheorem{mycor}{Corollary}
\newtheorem{mydef}{Definition}

\newcommand{\colorA}{white}
\newcommand{\colorB}{black}
\newcommand{\colorAsm}{w}
\newcommand{\colorBsm}{b}
\newcommand{\poly}{$\mathscr{P}$}
\newcommand{\faceset}{F\left(\mathscr{P}\right)}
\newcommand{\spc}{ }
\newcommand{\xj}{$x^j$}
\newcommand{\xk}{$x^k$}
\newcommand{\Sjk}{$S_{jk}$}
\newcommand{\Skj}{$S_{kj}$}

\DeclareMathOperator{\diag}{diag}


\begin{document}
\maketitle

\begin{abstract}
The Building Game is a sequential coloring process on polyhedra. We enumerate the Building Game state space for all polyhedra in the Platonic, Archimedean, and Catalan solids classes of up to 30 faces. By putting a probability distribution on each step of the Building Game process, a distribution is induced on the entire state space. With the help of a finite group theoretic identity, we find the explicit form of these distributions. Finally, we examine the properties of the resulting distributions.
\end{abstract}

\section{Introduction}
The Building Game (BG) was first considered by Zlotnick~\cite{Zlotnick1994} as a model for the assembly of polyhedral viral capsids. We formalize the idea as a sequential coloring process that progresses from a polyhedron \poly\spc with each face colored \colorA, through a number of intermediate states each having a mix of \colorA\spc and \colorB\spc faces, and ending with all of the faces colored \colorB. 
\begin{mydef}
A Building Game \textbf{intermediate} $x$ is a function from the faces of \poly\spc, $F\left(\mathscr{P}\right)$, to a color in $\left\{\colorA,\colorB\right\}$ such that the set $\left\{f_m \in F\left(\mathscr{P}\right) : x\left(f_m\right) = \colorB\right\}$ is edge connected along with the equivalence relation $x \sim x'$ if there is an element $g$ of \poly's rotation group $G$ that satisfies $x(f_m) = x'(g.f_m)$ for every $f_m \in F(\mathscr{P})$. 
\end{mydef}
For ease of exposition, we use the notational shorthand $\left(x\right)_m$ for $x\left(f_m\right)$ and $x = g.x'$ when $x(f_m) = x'(g.f_m)$ for every $f_m \in F(\mathscr{P})$. Additionally, we denote the intermediate satisfying $\left(x\right)_m = \colorA$ for all $f_m \in \faceset$ as $x^\colorAsm$ and similarly $x^\colorBsm$ is the intermediate with $\left(x\right)_m = \colorB$ for all $f_m \in \faceset$. The function counting the number of \colorB\spc faces an intermediate has is denoted $h\left(x\right) \doteq |\left\{f_m \in \faceset : \left(x\right)_m = \colorB\right\}|$.

\begin{mydef}
Two intermediates $x^j$ and $x^k$ are \textbf{connected} ($x^j \leftrightarrow x^k$) if $\left(x^j\right)_m = \left(x^k\right)_m$ for all $f_m \in \faceset$ except for exactly one face $f_n$ that has  $\left(x^j\right)_n \neq \left(x^k\right)_n$.
\end{mydef}
\begin{mydef}
A Building Game \textbf{pathway} is a sequence of intermediates $x^{p_0}, x^{p_1}, x^{p_2}, \dots, x^{p_N}$ such that $x^{p_0} = x^\colorAsm$, $x^{p_N} = x^\colorBsm$, $x^{p_i}$ is connected to $x^{p_{i+1}}$ and $h\left(x^{p_i}\right) = i$.
\end{mydef}

In this way it is useful to think of intermediates as connected if it is possible to color one face of the first intermediate to get the second and a pathway as a sequence of these connections between $x^\colorAsm$ and $x^\colorBsm$. Figure~\ref{fig:DodecBG} shows a Building Game pathway for the dodecahedron using Schlegel diagrams. The pathways has 13 intermediates since there must be exactly one intermediate $x^{p_i}$ satisfying $h\left(x^{p_i}\right) = i$ for each $i = 0,1,2,\dots,12$.

\begin{figure}[ht]
\caption{One Building Game pathway on the dodecahedron.}
\label{fig:DodecBG}
\end{figure}

With many pairs of connected intermediates, we organize these relations in a graph.

\begin{mydef}
The Building Game \textbf{state space} for a polyhedron \poly\spc is a graph in which the nodes are \poly's intermediates and a graph edge exists between two intermediates if and only if they are connected. 
\end{mydef}

When the intermediates are partitioned by their value of $h$, it is natural to arrange the state space as a tiered graph according to this partition. Figure~\ref{fig:CubeSS} shows the Building Game state space for the cube. As seen, each tier has intermediates with the same number of \colorB faces and connections thus exist with intermediates that are either in the tier directly above or below them. We can also see that there are three distinct pathways contained in the state space. 

\begin{figure}[ht]
\caption{The Building Game state space of the cube.}
\label{fig:CubeSS}
\end{figure}

Interestingly, it is not the case that the recoloring of each face of \xj\spc results in a distinct intermediate. 
\begin{mydef}
The number of different faces $\left|\left\{f_m \in \faceset : x^j + e^m \in \left[x^k\right]\right\}\right|$ of \xj\spc that can be colored to form \xk\spc is called the \textbf{degeneracy number} \Sjk.
\end{mydef}
It is important to note that in general the degeneracy number is not symmetric, i.e. \Sjk$\neq$\Skj\spc for some connections \xj\spc$\leftrightarrow$\spc\xk\spc in the state space. Both figures~\ref{fig:DodecBG} and ~\ref{fig:DodecBG} show the forward and backward degeneracy numbers for each connection.

\subsection{Related Work}
--Like polyominos on polyhedra

\subsection{Applications}
--Viral capsid assembly

--Self-assembly of molecular cages

--Self assembly for manufacturing purposes
%~\cite{Endres2005}
\subsection{Paper Overview}
-- Summary of subsequent sections

\section{Enumerative Results}

As we consider polyhedra with more and more faces, there is a combinatorial explosion in the number intermediates in state space. While the 6-faced cube state space has only 8 nodes and 9 nodes, the 20-faced icosahedron state space has 2,649 nodes and 17,241 nodes and the 26-faced truncated cuboctahedron state space has 1,525,605 nodes and 17,672,377. Figure \ref{fig:bgtable} details state space sizes of all polyhedra in the Platonic, Archimedean, and Catalan solid classes of up to 26 faces. 


Also something about pathway statistics. 


\begin{figure}[ht]
\scalebox{0.6}{
%{\footnotesize
\begin{tabular}{ l | c | c | c | c || r | r | r}
Polyhedra Name & Class & F$\left(\mathscr{P}\right)$ & E$\left(\mathscr{P}\right)$ & V$\left(\mathscr{P}\right)$ & Intermediates & Connections & Pathways \\
  \hline    
Tetrahedron                     & P & 4 & 6 & 4         & 5     	& 4             & 1\\
Cube                            & P & 6 & 12 & 8        & 9     	& 10    	& 3\\
Octahedron                      & P & 8 & 12 & 6        & 15    	& 22    	& 14\\
Dodecahedron                    & P & 12 & 30 & 20      & 74    	& 264   	& 17,696 \\
Icosahedron                     & P & 20 & 30 & 12      & 2,650 	& 17,242        & 57,396,146,640\\
Truncated Tetrahedron           & A & 8 & 18 & 12       & 29    	& 65            & 402\\
Cuboctahedron                   & A & 14 & 24 & 12 	& 341   	& 1,636         & 10,170,968\\
Truncated Cube                  & A & 14 & 36 & 24 	& 500   	& 2,731         & 101,443,338 \\
Truncated Octahedron            & A & 14 & 36 & 24 	& 556           & 3,071         & 68,106,377\\
Rhombicuboctahedron             & A & 26 & 48 & 24 	& 638,851       & 6,459,804     & 16,494,392,631,838,879,380\\
Truncated Cuboctahedron         & A & 26 & 72 & 48 	& 1,525,605     & 17,672,377    & ? \\
Icosidodecahedron               & A & 32 & 60 & 30 	& ?             & ?             & ?\\
Truncated Dodecahedron          & A & 32 & 90 & 60 	& ?             & ? 	        & ? \\
Truncated Icosahedron           & A & 32 & 90 & 60 	& ?             & ? 	        & ?\\
Triakis Tetrahedron             & C & 12 & 18 & 8       & 99            & 319           & 38,938\\
Rhombic Dodecahedron            & C & 12 & 24 & 14 	& 128           & 494           & 76,936\\
Triakis Octahedron              & C & 24 & 36 & 14 	& 12,749        & 81,297        & 169,402,670,046,670\\
Tetrakis Hexahedron             & C & 24 & 36 & 14 	& 50,768        & 394,278       & 4,253,948,297,210,346\\
Deltoidal Icositetrahedron      & C & 24 & 48 & 26 	& 209,676       & 1,989,549     & ? \\
Pentagonal Icositetrahedron     & C & 24 & 60 & 38 	& 345,939       & 3,544,988     & 2,828,128,000,716,774,492\\
Rhombic Triacontahedron         & C & 30 & 60 & 32 	& ?             & ?             & 5,266,831,101,345,821,968\\
  \hline  
\end{tabular}
}
\caption{Table of polyhedra in the Platonic (P), Archimedean (A), and Catalan (C) solid classes of up to 32 faces and their Building Game state space statistics.}
\label{fig:bgtable}
\end{figure}


\begin{figure}[ht]
\scalebox{0.6}{
%{\footnotesize
\begin{tabular}{ l | c | c | c | c || r | r | r}
Polyhedra Name & Class & F$\left(\mathscr{P}\right)$ & E$\left(\mathscr{P}\right)$ & V$\left(\mathscr{P}\right)$ & Intermediates & Connections & Pathways \\
  \hline    
Tetrahedron                     & P & 4 & 6 & 4         & 5     & 4 & 1\\
Cube                            & P & 6 & 12 & 8        & 8     & 8 & 2\\
Octahedron                      & P & 8 & 12 & 6        & 12    & 12 & 14 \\
Dodecahedron                    & P & 12 & 30 & 20      & 53    & 156 & 2166\\
Icosahedron                     & P & 20 & 30 & 12      & 468   & 1984 & 105999738\\
Truncated Tetrahedron           & A & 8 & 18 & 12       & 22	& 42 & 174\\
Cuboctahedron                   & A & 14 & 24 & 12 	& 137	& 470 & 477776\\
Truncated Cube                  & A & 14 & 36 & 24 	& 248	& 1002 & 5232294\\
Truncated Octahedron            & A & 14 & 36 & 24 	& 343	& 1466 & 5704138\\
Rhombicuboctahedron             & A & 26 & 48 & 24 	& 70836	& 462149 &  48399693494788840\\
Truncated Cuboctahedron         & A & 26 & 72 & 48 	& ?	& ? & ?\\
Icosidodecahedron               & A & 32 & 60 & 30 	& ?	& ? & ?\\
Truncated Dodecahedron          & A & 32 & 90 & 60 	& ?	& ? & ?\\
Truncated Icosahedron           & A & 32 & 90 & 60 	& ?	& ? & ?\\
Triakis Tetrahedron             & C & 12 & 18 & 8       & 49	& 116 & 5012\\
Rhombic Dodecahedron            & C & 12 & 24 & 14 	& 68	& 196 & 6258\\
Triakis Octahedron              & C & 24 & 36 & 14 	& 667	& 2383 & 15255459\\
Tetrakis Hexahedron             & C & 24 & 36 & 14 	& 4220	& 21079 & 5854799360107\\
Deltoidal Icositetrahedron      & C & 24 & 48 & 26 	& ?	& ? & ?\\
Pentagonal Icositetrahedron     & C & 24 & 60 & 38 	& 95127	& 654537 & 5607231936129109\\
Rhombic Triacontahedron         & C & 30 & 60 & 32 	& 97368	& 697623 & 6889989896241902854\\
  \hline  
\end{tabular}
}
\caption{Table of polyhedra in the Platonic (P), Archimedean (A), and Catalan (C) solid classes of up to 32 faces and their Building Game state space shellability statistics.}
\label{fig:bgtable_shell}
\end{figure}


\subsection{Bounds and Asymptotics}
Have upper, but what about lower? 
\subsection{Methods}
\section{The Building Game as a Stochastic Process}
\label{sec:Prob}
Since the Building Game is a sequential process with several choices at each step, it is natural to consider it as a stochastic process. By putting a distribution on all possible faces that can be colored \colorB\spc at each step of the Building game, a distribution on the space of pathways is implicitly defined. Thus, for a choice of this transition rule, we can ask questions about the likelihood of the different pathways. 

--Math and graphical results about putting a distribution on pathways

\subsection{Forward and Backward Transitions}

If we allow faces be changed both from \colorA\spc to \colorB\spc and from \colorB\spc to \colorA, the process consists of transitions from intermediate to intermediate along state space connections. By specifying a distribution on these transitions, it will induce a stationary measure on the state space.  

We define the Markov process $X_t$ by the transition rate matrix $Q$, with the heuristic that the rate of transition to an intermediate \xk from an intermediate \xk should be proportional to the number of faces of \xj that can be colored to reach the intermediate \xk. For this reason, we include the degeneracy number \Sjk\spc as a factor in the transition rate matrix. Furthermore, we model the process after and energetic model in which each intermediate has an energy and to transtion between intermediates, an energy barrier $E_{jk} = E_{kj}$ must be overcome. 
%\begin{align}
%\label{eq:TransitionProbability}
% P_{jk} = \frac{1}{z_j}S_{jk}\rh 
%\end{align}
\begin{align}
\label{eq:TransitionRate}
Q_{jk} &= S_{jk}e^{-\beta\left(E_{jk} - E_j\right)} \\
Q_{jj} &= -z_j \\
\end{align}
Here, $z_j \doteq \sum_{\ell: \ell \neq j} S_{j\ell}e^{-\beta\left(E_{j\ell} - E_j\right)}$ is the rate at which the process leaves \xj. 

\begin{mythm}
\label{thm:StatDist}
If the transition rate matrix $Q$ can be decomposed as $Q = DC$ where $D$ is diagonal with each entry of the diagonal positive and $C$ is a non-negative symmetric matrix with $C_{jk} > 0$ if and only if $x^j$ and $x^k$ are connected, then $X_t$ has the unique stationary distribution $\pi = \diag\left(D^{-1}\right)$.         
\end{mythm}
\begin{proof}
First, we show $Q$ and $\pi$ satisfy detailed balance.
\begin{align}
\pi_jQ_{jk} &= \left(\frac{1}{D_{jj}}\right)\left(D_{jj}C_{jk}\right) \\
&= C_{jk} \\
&= C_{kj} \\
&= \left(\frac{1}{D_{kk}}\right)\left(D_{kk}C_{kj}\right) \\
                    &= \pi_kQ_{kj}
\end{align}

-- Prove aperiodicity 
-- Prove positive reccurence

\end{proof}

In order to use theorem~\ref{thm:StatDist} to find the stationary distribution for the transition rule~\ref{eq:TransitionProbability}, we must be able to decompose the degeneracy number \Sjk\spc to fit the template of $\mathbf{C}$ and $\mathbf{D}$. In the following section we derive group theoretic identities to show that this is possible.

\subsection{Hitting Times}

\begin{align}
	\tau^{A}_{j} &\doteq \inf\left\{t \geq 0 : X_t \in A, X_0 = x^j\right\}
\end{align}

\begin{align}
	\nu^{A}_{j} &\doteq \inf\left\{n \geq 0 : Y_n \in A, Y_0 = x^j\right\}
\end{align}

For $j \not\in A$.
\begin{align}
	E\left[\tau^{A}_{j}\right] &= E\left[E\left[\tau^{A}_{j} | Y_1 \right]\right] \\
        &= E\left[ Exp\left(z_j\right) + \tau^{A}_{Y_1} \right] \\
        &=  \frac{1}{z_j} + E\left[\sum_{k}\tau^{A}_{Y_1}\mathbbm{1}_{Y_1 = k}\right] \\
        &=  \frac{1}{z_j} + \sum_{k: k\neq j}E\left[\tau^{A}_{k}\right] P\left(Y_1 = k\right) \\
        &=  \frac{1}{z_j}\left(1 + \sum_{k: k\neq j}q_{jk}E\left[\tau^{A}_{k}\right]\right)     \\
  \sum_{k}q_{jk}E\left[\tau^{A}_{k}\right] &= 1 \\
\end{align}

For $j \in A$.
\begin{align}
	E\left[\tau^{A}_{j}\right] &= 0 \\
\end{align}

As a linear system:
\begin{align}
	\left(\diag\left(\mathbbm{1}_A\right) - \diag\left(\mathbbm{1}_{A^c}\right)Q\right)E\left[\tau^{A}\right] =\mathbbm{1}_{A^c}\\
\end{align}


\begin{align}
\psi_j^A\left(t\right) &\doteq P\left(\tau^A_j \leq t\right) \\
\psi_j^A\left(0\right) &= \mathbbm{1}_{j\in A} \\
\psi_j^A\left(t\right) &= 0 \forall j \in A \\                       
\end{align}

For $j \not\in A$.

\begin{align}
\psi_j^A\left(t\right) &\doteq P\left(\tau^A_j \leq t\right) \\
                       &= \sum_k P\left(\tau^A_j \leq t | Y_1 = x^k\right) P\left(Y_1 = x^k\right) \\ 
                       &= \frac{1}{z_j}\sum_{k: k \neq j} q_{jk} P\left(Exp\left(z_j\right)\tau^A_j \leq t\right)  \\
                       &= \frac{1}{z_j}\sum_{k: k \neq j} q_{jk} \int^t_0 P\left(\tau^A_j \leq t - s\right) z_j e^{-z_j s} ds  \\
                       &= \sum_{k: k \neq j} q_{jk} \int^t_0\psi^A_k\left(t-s\right)e^{-z_j s} ds  \\
                       &= \sum_{k: k \neq j} q_{jk} \int^t_0\psi^A_k\left(r\right)e^{-z_j\left(t-r\right)} dr  \\
e^{z_jt}\psi^A_j\left(t\right) &= \sum_{k: k \neq j} q_{jk} \int^t_0 e^{z_jr}\psi^A_k\left(r\right) dr  \\
e^{z_jt}\frac{d\psi^A_j}{dt} + z_j e^{z_j t} \psi^A_j\left(t\right) &= \sum_{k: k \neq j} q_{jk} e^{z_jt}\psi^A_k\left(t\right)  \\
\frac{d\psi^A_j}{dt} &= \sum_{k} q_{jk} \psi^A_k\left(t\right) 
\end{align}

Combining both cases, we get the linear system and solution.

\begin{align}
        \frac{d\psi^A}{dt} &= \diag\left(\mathbbm{1}_{A^c}\right)Q\psi^A \\
        \psi^A\left(0\right) &= \mathbbm{1}_{A} \\
        \psi^A\left(t\right) &= e^{\diag\left(\mathbbm{1}_{A^c}\right)Qt} \mathbbm{1}_{A} \\ 
\end{align}

This is the solution for the CDF of the stopping time $\tau^A$, but we can also compute the PDF explicitly for $t > 0$.

\begin{align}
        p\left(\tau^A = t\right) &= \frac{d\psi^A}{dt} \\
        &= \diag\left(\mathbbm{1}_{A^c}\right)Q\psi^A
\end{align} 


\section{A Finite Geometric Result}

Since we define Building Game intermediates as rotationally unique from each other, it is useful to think about the problem in the context of $\mathscr{P}$'s rotational symmetry group $G \doteq G\left(\mathscr{P}\right)$ and group actions. For an intermediate $x^j$, the number of symmetries $r_j$ is the order of the stabilizer subgroup $G_{x^j} \doteq \left\{g \in G : g.x^j = x^j\right\}$ of $G$ that fixes $x^j$. Suppose $x^j$ and $x^k$ are connected in the state space and $\varphi$ is one of the $S_{jk}$ faces that, when added to $x^j$, forms $x^k$. We say $x^j + \varphi = x^k$. The degeneracy number $S_{jk}$ can then be expressed as the order of the orbit $\left(G_{x^j}\right).\varphi$ of $\varphi$ with respect to $x^j$'s stabilizer subgroup. Analogously, we define the reverse degeneracy number as $S_{kj} \doteq \left|\left(G_{x^k}\right).\varphi\right|$

\begin{mylem}
\label{lem:I}
For Building Game intermediates $x^j$ and $x^k$ connected in the state space and a face $f_m \in \faceset$ satisfying $x^j + e^m = x^k$, the stabilizer subgroup $G_{x^j,e^m}$ that fixes both $x^j$ and $e^m$ is the same stabilizer subgroup $G_{x^k,e^m}$ that fixes $x^k$ and $e^m$.
\end{mylem}
\begin{proof}
\begin{align}
G_{x^j,e^m} &\doteq \left\{g \in G | g.x^j = x^j, g.e^m = e^m \right\} \\
                &= \left\{g \in G | g.\left(x^k - e^m\right) = x^k - e^m, g.e^m = e^m \right\} \\
                &= \left\{g \in G | g.x^k = x^k, g.e^m = e^m \right\} \\
                &\doteq G_{x^k,e^m}
\end{align}
\end{proof}


\begin{mythm}
\label{thm:J}
For two Building Game intermediates $x^j$ and $x^k$ are connected in the BG state space, $r_kS_{jk} = r_jS_{kj}$.
\end{mythm}
\begin{proof}
Let $e^m$ be a face such that $x^k = x^j + e^m$. Then, by the orbit-stabilizer theorem, Lagrange's Theorem and lemma~\ref{lem:I} we have the following~\cite{Rotman1995}.
\begin{align}
\frac{r_j}{S_{jk}} &\doteq \frac{\left|G_{x^j}\right|}{\left|\left(G_{x^j}\right).e^m\right|} \\
                   &= \left[G_{x^j} : \left(G_{x^j}\right).e^m \right] \\
                   &= \left|G_{x^j,e^m}\right| \\
                   &= \left|G_{x^k,e^m}\right| \\
                   &= \left[G_{x^k} : \left(G_{x^k}\right).e^m \right] \\
                   &= \frac{\left|G_{x^k}\right|}{\left|\left(G_{x^k}\right).e^m\right|} \\
                   &\doteq \frac{r_k}{S_{kj}} 
\end{align}
The result $r_kS_{jk} = r_jS_{kj}$ follows.
\end{proof}

\section{Stationarity}

\begin{mythm}
\label{thm:E}
The Markov process $X_t$ defined by the transition rate matrix $Q$ in equation~\ref{eq:TransitionRate} admits the unique stationary distribution $\frac{1}{zr_j}e^{-\beta E_j}$ where $z \doteq \sum_\ell \frac{1}{r_\ell}e^{-\beta E_\ell}$ is the partition function. 
\end{mythm}
\begin{proof}
We take $C_{jk} \doteq \frac{S_{jk}}{zr_j}e^{-\beta E_{jk}}$ and notice that it is symmetric by theorem~\ref{thm:J}. With $D_{jj} \doteq zr_je^{\beta E_j}$ we have our partition.  
\begin{align}
Q_{jk} &= S_{jk}e^{-\beta\left(E_{jk} - E_j\right)} \\
       &= \left(zr_je^{\beta E_j}\right) \left(\frac{S_{jk}}{zr_j}e^{-\beta E_{jk}}\right) \\
       &= D_{jj}C_{jk}   
\end{align}
Thus, by theorem~\ref{thm:StatDist},  $\pi_j = \frac{1}{D_{jj}} = \frac{1}{zr_j}e^{-\beta E_j}$.
\end{proof}


\section{Discussion}
\subsection{Nonenumerative Approaches}

\section*{Acknowledgments}\label{ackowledgements}
        Supported by NSF grants DMS 07-48482 and EFRI 10-22638

\section{Potential Citations (temp)}
~\cite{Coxeter1963}
~\cite{Grunbaum2003}
~\cite{Cromwell1997}
~\cite{Ziegler1995}
~\cite{Gidas1995}
~\cite{Eden1961}
~\cite{Grayson2012}


\bibliographystyle{plain}
%\nocite{*} 
\bibliography{Master}


%\appendix
%\section{Proof of Theorem ???}
%\section{Proof of Theorem ???}

\end{document}


%%%%%%%%%%%%%%%%%%%% CCCCUUUUTTTT %%%%%%%%%%%%%%%%%%%%%%%%%%%%%%%%%%%%

\subsection{A Curious Observation}

As part of the exploration of the nature of the degeneracy numbers $S_{jk}$ to find a stationary distribution, an interesting phenomena was noticed. Given a closed path $x^{k_0} \to x^{k_1} \to \cdots \to x^{k_n}$ through the state space, starting at $x^{k_0}$ and ending back at $x^{k_n} \doteq x^{k_0}$, the the identity given in equation~\ref{eq:DegenIdent} was observed to hold independent of the particular closed path. 
\begin{align}
\prod_{i=1}^n \frac{S_{k_i,k_{i-1}}}{S_{k_{i-1},k_i}} = 1
\label{eq:DegenIdent}
\end{align}
At first this relation between ratios of degeneracy numbers along the path was noticed in the cube's state space and it can easily be seen by examining figure~\ref{fig:CubeSS}. To test this conjecture more fully, it was subsequently verified on our library of state spaces for other polyhedra. This exploration lead to the discovery of a more general relation between intermediates that it described below.

In an attempt to get the the heart of the relation in equation~\ref{eq:DegenIdent}, we compiled a list of different geometric and combinatorial statistics for each state space connection $x_j \leftrightarrow x_k$. These statistics included number of faces in $x_j$ and $x_k$, the order of $x_j$ and $x_k$'s rotation groups $r_j$ and $r_k$, and a few others. With the ansatz that there may be a multiplicative relation between the degeneracy number $S_{jk}$ and some of these $d$ statistics $Z_{jk}^{\left(i\right)} \in \mathbbm{Z}^d$ for each connection, the following linear regression model was used.
\begin{align}
\log\left(S_{jk}\right) = \sum_{i=1}^d \beta^{\left(i\right)}\log\left(Z_{jk}^{\left(i\right)}\right)
\label{eq:LinReg}
\end{align}
The regression coefficients $\boldsymbol{\beta} \in \mathbbm{R}^d$ that minimize the least-squares residual were solved for and it was found that a perfect relation was found in the sense that equation~\ref{eq:LinReg} held exactly for all connections tested. The three non-zero entries in $\boldsymbol{\beta}$ corresponded to $r_j$, $r_k$ and $S_{kj}$. This result meant that $\log\left(S_{jk}\right) = \log\left(S_{kj}\right) + \log\left(r_{j}\right) - \log\left(r_{k}\right)$, or 
\begin{align}
S_{jk} = \frac{r_jS_{kj}}{r_k}
\label{eq:RSConjecture}
\end{align} 
for all connections we tested. This leads to the obvious conjecture that equation~\ref{eq:RSConjecture} holds for any connection in any Building Game state space. 

In fact, if this is the case, it would explain why equation~\ref{eq:DegenIdent} holds. 
\begin{align}
        \prod_{i=1}^n \frac{S_{k_i,k_{i-1}}}{S_{k_{i-1},k_i}} &= \prod_{i=1}^n \frac{r_{k_i}}{r_{k_{i-1}}} \\ 
        &= \frac{r_{k_n}}{r_{k_{0}}} \\ 
        &= \frac{r_{k_0}}{r_{k_{0}}} \\
        &= 1 
\end{align}



\begin{figure}[ht]
\scalebox{0.6}{
%{\footnotesize
\begin{tabular}{ l | c | c | c | c || r | r | r}
Polyhedra Name & Class & F$\left(\mathscr{P}\right)$ & E$\left(\mathscr{P}\right)$ & V$\left(\mathscr{P}\right)$ & Intermediates & Connections & Pathways \\
  \hline    
Tetrahedron & P & 4 & 6 & 4 & 5 & 4 & 1\\
Cube & P & 6 & 12 & 8 & 9 & 10 & 3\\
Octahedron & P & 8 & 12 & 6 & 15 & 22 & 14\\
Dodecahedron & P & 12 & 30 & 20 & 74 & 264 & 17,696 \\
Icosahedron & P & 20 & 30 & 12 & 2,650 & 17,242 & 57,396,146,640\\
Truncated Tetrahedron & A & 8 & 18 & 12 & 29 & 65 & 402\\
Cuboctahedron & A & 14 & 24 & 12 & 341 & 1,636 & 10,170,968\\
Truncated Cube & A & 14 & 36 & 24 & 500 & 2,731 & 101,443,338 \\
Truncated Octahedron & A & 14 & 36 & 24 & 556 & 3,071 & 68,106,377\\
Rhombicuboctahedron & A & 26 & 48 & 24 & 638,851 & 6,459,804 & 16,494,392,631,838,879,380\\
Truncated Cuboctahedron & A & 26 & 72 & 48 & 1,525,605 & 17,672,377 & ? \\
Icosidodecahedron & A & 32 & 60 & 30 & ? & ? & ?\\
Truncated Dodecahedron & A & 32 & 90 & 60 & ? & ? & ? \\
Truncated Icosahedron & A & 32 & 90 & 60 & ? & ? & ?\\
Triakis Tetrahedron & C & 12 & 18 & 8 & 99 & 319 & 38,938\\
Rhombic Dodecahedron & C & 12 & 24 & 14 & 128 & 494 & 76,936\\
Triakis Octahedron & C & 24 & 36 & 14 & 12,749 & 81,297 & 169,402,670,046,670\\
Tetrakis Hexahedron & C & 24 & 36 & 14 & 50,768 & 394,278 & 4,253,948,297,210,346\\
Deltoidal Icositetrahedron & C & 24 & 48 & 26 & 209,676 & 1,989,549 & ? \\
Pentagonal Icositetrahedron & C & 24 & 60 & 38 & 345,939 & 3,544,988 & 2,828,128,000,716,774,492\\
Rhombic Triacontahedron & C & 30 & 60 & 32 & ? & ? & 5,266,831,101,345,821,968\\
  \hline  
\end{tabular}
}
\caption{Table of polyhedra in the Platonic (P), Archimedean (A), and Catalan (C) solid classes of up to 32 faces and their Building Game state space statistics.}
\label{fig:bgtable}
\end{figure}






%%%%%%%


The process begins with each face of a polyhedron \poly\spc all colored the same color, say \colorA. A face is then chosen and its color is changed to a second color, \colorB. From there, at each step a \colorA\spc face that is edge-adjacent to a \colorB\spc face is chosen and recolored \colorB. The process continues until all of \poly's faces are \colorB. We show one instance of the Building Game process for the dodecahedron in figure~\ref{fig:DodecBG}. 

Each possible coloring of the polyhedron from the Building Game is referred to as an \textbf{intermediate}. Since we assume each face of the same color is otherwise indistinguishable, there is a rotational equivalence class on intermediate such that two intermediates are equivalent if the first is a rotation of the second. 

\begin{figure}[ht]
\caption{One instance of the Building Game on the dodecahedron.}
\label{fig:DodecBG}
\end{figure}

We define the \textbf{state space} of the Building Game to be a graph in which the nodes are all of the possible intermediates allowed by the BG. Connections exist between two intermediates \xj\spc and \xk\spc if it is possible to color a single face of \xj\spc to form \xk. Interestingly, it is not the case that the recoloring of each face of \xj\spc maps to a distinct intermediate. Thus, the number of different faces of \xj\spc that can be colored to form \xk\spc is called the \textbf{degeneracy number} \Sjk. It is important to note that in general the degeneracy number is not symmetric, i.e. \Sjk$\neq$\Skj\spc for some connections \xj\spc$\leftrightarrow$\spc\xk\spc in the state space. 


Since the state space can be partitioned by the number of \colorB\spc faces each intermediate has, it is natural to view the state space as a tiered graph in which intermediates in each tier have the same number of \colorB\spc faces. Organized this way, intermediates can only connect to those in the tier above or below them. A \textbf{pathway} in the state space is a sequence of intermediates $x^{k_0} \to x^{k_1}\to \cdots \to x^{k_F}$ connected in the state space such that $x^{k_0}$ is the intermediate with all \colorA\spc faces, $x^{k_F}$ is the intermediate with all \colorB\spc faces, and $x^{k_m}$ has $m$ \colorB\spc faces.  



%%%%%%%%


To find the stationary measure of this process, we check if it is possible for our transition measure to satisfy the detailed balance equation $\pi_jP_{jk} = \pi_kP_{kj}$. As seen in equation~\ref{eq:DBTry}, for detailed balance to be satisfied, we must find a way to separate the ratio $\frac{S_{jk}}{S_{kj}}$ into a part depending only on $j$ terms and a part depending only on $k$ terms. 
\begin{align}
        \frac{\pi_k}{\pi_j} &= \frac{P_{jk}}{P_{kj}} \\
        &= \frac{S_{jk}\rho_{jk}z_k}{S_{kj}\rho_{kj}z_j} \\
        &= \frac{z_k}{z_j}\frac{S_{jk}}{S_{kj}} \label{eq:DBTry}
\end{align}
While detailed balance is not a necessary condition for the existence of a stationary distribution, along with positive recurrence of the process, it is sufficient and would provide the exact form of the necessarily unique stationary distribution.


%%%%%%%%%%


As stated in section~\ref{sec:Prob}, if we can find a distribution $\pi$ such that the detailed balance equation $\pi_jP_{j,k} = \pi_kP_{k,j}$ is satisfied, we know that $\pi$ is the unique stationary distribution under $P_{jk}$. With theorem~\ref{thm:J} we find that $\frac{S_{jk}}{S_{kj}} = \frac{r_j}{r_k}$ is indeed separable as conjectured in equation~\ref{eq:DBTry}. This allows us to find the stationary distribution. 

\begin{mythm}
\label{thm:E}
The Markov chain $X_t$ defined by the transition rule $P_{jk}$ in equation~\ref{eq:TransitionProbability} admits the unique stationary distribution $\pi_j = \frac{1}{z}\left(\frac{z_j}{r_j}\right)$ where $z \doteq \sum_i \frac{z_i}{r_i}$ is the partition function. 
\end{mythm}
\begin{proof}
It suffices to show detailed balance and that each state is positively recurrent. By our definitions of $\pi$ and $P$ along with Theorem~\ref{thm:J}, we find detailed balance. 
\begin{align}
        \pi_jP_{jk} &= \frac{z_j}{z r_j}\frac{1}{z_j}S_{jk}\rho_{jk} \\ 
                  &= \frac{1}{z}\left(\frac{S_{jk}}{r_{j}}\right)\rho_{jk} \\
                  &= \frac{1}{z}\left(\frac{S_{kj}}{r_k}\right)\rho_{kj} \\
                   &= \frac{z_k}{z r_k}\frac{1}{z_k}S_{kj}\rho_{kj} \\
                    &= \pi_kP_{kj}
\end{align}
Clearly $X_t$ is positive recurrent since it is a finite irreducible Markov chain on a connected state space in which there exist forward and backward transitions between each connected node with positive probabilities.
\end{proof}

