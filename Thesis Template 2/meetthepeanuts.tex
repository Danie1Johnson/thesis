\chapter{Meet the Peanuts}\label{FAQ}



\section{Who are they?}

Snoopy\index{Snoopy} is an extroverted beagle with a Walter Mitty complex. He is a virtuoso at every endeavor- at least in his daydreams atop his doghouse. He regards his master, Charlie Brown, as ``that round-headed kid'' who brings him his supper dish. He is fearless though prudently cautious about ``the cat next door.'' He never speaks- that would be one human trait too many- but he manages to convey everything necessary in facial expressions and thought balloons. A one-man show with superior intelligence and vivid imagination, he has created such multiple personalities as: Joe Cool, World War I Flying Ace, Literary Ace, Flashbeagle, Vulture, Foreign Legionnaire, etc.


Charlie Brown wins your heart with his losing ways. It always rains on his parade, his baseball game, and his life. He's an inveterate worrier who frets over trifles (but who's to say they're trifles?). Although he is concerned with the true meaning of life, his friends sometimes call him ``blockhead.'' Other than his knack for putting himself down, there are few sharp edges of wit in his repertoire; usually he's the butt of the joke, not the joker. He can be spotted a mile away in his sweater with the zig zag trim, head down, hands in pocket, headed for Lucy's psychiatric booth. He is considerate, friendly and polite and we love him knowing that he'll never win a baseball game or the heart of the little red-haired girl, kick the football Lucy is holding or fly a kite successfully. His friends call him ``wishy-washy,'' but his spirit will never give up in his quest to triumph over adversity.

Woodstock is the smallest of the Peanuts characters but has a big presence for a little bird. He's a little inept, his flying and logic are erratic, but he can type and take shorthand and usually is game for anything Snoopy\index{Snoopy} wants to do. Although he's the butt of many of Snoopy\index{Snoopy}'s practical jokes, he's the beagle's closest friend and confidant- and has made attempts at retaliation. Because of his size and the company he keeps, Woodstock is an accident waiting to happen. Being a bird and tiny, he gets a little insecure around Thanksgiving and big moving objects. He's the only baseball player who gets an automatic walk if the ball rolls over him. Woodstock talks birdspeak only, and finds an alphabet made up entirely of exclamation points quite adequate to express such emotions as distress, frustration and a real temper. His flocking friends are Bill, Harriet, Olivier and Conrad.

Linus Van Pelt inspired the term ``security blanket'' with his classic pose. He is the intellectual of the gang, and flabbergasts his friends with his philosophical revelations and solutions to problems. He suffers abuse from his big sister, Lucy, and the unwanted attentions of Charlie Brown's little sister, Sally. He is a paradox: despite his age, he can put life into perspective while sucking his thumb. He knows the true meaning of Christmas while continuing to believe in the Great Pumpkin.


Lucy Van Pelt works hard at being bossy, crabby and selfish. She is loud and yells a lot. Her smiles and motives are rarely pure. She's a know-it-all who dispenses advice whether you want it or not--and for Charlie Brown, there's a charge. She's a fussbudget, in the true sense of the word. She's a real grouch, with only one or two soft spots, and both of them may be Schroeder, who prefers Beethoven. As she sees it, hers is the only way. The absence of logic in her arguments holds a kind of shining lunacy. When it comes to compliments, Lucy only likes receiving them. If she's paying one--or even smiling--she's probably up to something devious.


Sally Brown's brother, Charlie Brown, was so pleased and proud when she was born that he passed out chocolate cigars. Since then he's been trying to understand her. She always looks for the easy way out, particularly at school, where her view of life reflects much of the frustration and confusion kids experience. Her speech is riddled with malapropisms. Uninhibited, and precocious, she has a schoolgirl crush on Linus, her ``Sweet Babboo.'' She may never win Linus' heart, but she has her big brother wrapped around her little finger. Sally, writing letters or doing homework, causes pain and joy to her fans in roughly equal proportions.



Schroeder, who idolizes Beethoven, brought classical music to the Peanuts strip. Reserved and usually unruffled, Schroeder reacts only when Woodstock tries to make his grand piano into a playground, or Lucy seeks to make it her courting grounds. The latter can lead to minor violence.



Peppermint Patty is a pro on the baseball diamond, but in the classroom she's a D-minus all the way. Bold, brash and tomboyish, what she lacks in common sense she makes up for in sincerity. She's the only one who calls Charlie Brown ``Chuck.'' Oblivious to much that goes on around her, for a long time she seemed unaware that ``the funny-looking kid who plays shortstop'' was a beagle. She has trouble staying awake in class; most of her waking hours in the schoolroom are spent analyzing the probability patterns of true-false tests.


Marcie\index{Marcie} is Peppermint Patty's best friend. From the moment they met at summer camp, Marcie\index{Marcie} has called Peppermint Patty "Sir" out of admiration and misguided manners. An unlikely pair, they seem to have nothing in common yet that is what makes their friendship so genuine. Marcie\index{Marcie} is the smartest of the Peanuts gang, but also the most naive. She's always willing to help out her friend with school work and she's not above sharing test answers or calling her on the phone to remind her of homework assignments. There is an innocence to Marcie\index{Marcie} and Peppermint Patty is her protector. Marcie\index{Marcie} is also completely inept when it comes to sports, yet they still let her play on the baseball team. If Marcie\index{Marcie} and Peppermint Patty ever have a falling out it's likely to be over Charlie Brown, who they both secretly love.


Before grunge was cool, Pigpen\index{Pigpen} made his debut in the Peanuts comic strip on July 13, 1954 and since then has been the butt of "dirt" gags. He walks around in a cloud of dust, sprinkling dirt on all he comes in contact with. Pigpen\index{Pigpen} is happily messy. He doesn't try to explain it, hide it, fight it. For him, it's just a fact of life. His slovenly ways paid off in 1993 with a series of television commercials for Regina vacuum cleaners which combined animation with live-action.


Franklin met Charlie Brown at the beach in 1968. They'd never met before because they went to different schools, but they had fun playing ball so Charlie Brown invited Franklin to visit him at this house across town for another play session. Later, Franklin turned up as center-fielder on Peppermint Patty's baseball team and sits in front of her at school. Franklin is thoughtful and can quote the Old Testament as effectively as Linus. In contrast with the other characters, Franklin has the fewest anxieties and obsessions. He and Charlie Brown spend quite a bit of time talking about their respective grandfathers. When Franklin first appeared in the late 60s, his noticeably darker skin set some readers in search of a political meaning. However, the remarkable becomes unremarkable when readers learn that Schulz simply introduced Franklin as another character, not a political statement.


Rerun Van Pelt is often mistaken for Linus even though he's his little brother. He can always be recognized in his trademark overalls. Rerun is more skeptical than his brother, much harder to convince, and always gets around Lucy where Linus gives in. His only fear is being the passenger on one of his mother's bicycle-riding errands. Somehow, Rerun is the only witness to her riding into grates and potholes. Luckily, he always wears a helmet. Rerun also longs for a dog of his own, but since his parent won't let him have one, he tries to ``borrow'' Snoopy\index{Snoopy} from Charlie Brown. Snoopy\index{Snoopy} won't have any part of it unless Rerun brings cookies.



\section{Furthermore}

Peanuts was one of the first comic strips with more than two or three characters. Just like your own family and relatives, each Peanuts character brings special humor and insight to life.\footnote{Everything in this chapter was ungraciously reproduced from {\em http://www.unitedmedia.com/} to check formatting.  This is not at all intended for reproduction.}
